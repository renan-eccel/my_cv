\cvsection{Educação}

\cvevent{Mestrado em Engenharia de Automação e Sistemas}
        {UFSC}
        {Jan 2016 -- Nov 2019}
        {Florianópolis, SC}

Pesquisa na área de roteamento dinâmico de veículos, com ênfase em coleta
e avaliação dos dados de instâncias de benchmark presentes na literatura.
Para isso usou-se a linguagem Python e as bibliotecas pandas e seaborn.

\divider

\cvevent{Intercâmbio}
        {University of Florida}
        {Jan 2014 -- Dez 2014}
        {Gainesville, USA}
        
\divider

\cvevent{Bacharelado em Engenharia de Transporte e Logística}
        {UFSC}
        {Jan 2010 -- Dez 2015}
        {Joinville, SC}
        
        
        
        
        
\cvsection{Idiomas}

\cvskill{Portugês}{5}
\divider

\cvskill{Inglês}{4}
\divider

\cvskill{Espanhol}{3}






\cvsection{Competências}

\cvtag{Trabalho em equipe} 
\cvtag{Curiosidade}
\cvtag{Atenção aos detalhes}
\cvtag{Pragmatismo}
\cvtag{Aprendizado constante}

\divider

\cvtag{Otimização matemática}
\cvtag{Algoritmos}
\cvtag{Análise estatística}
\cvtag{Geoprocessamento}
\cvtag{Simulação de tráfego}


\divider

\cvtag{Python}
\cvtag{Java}
\cvtag{Julia}
\cvtag{\LaTeX}
\cvtag{QGIS}
\cvtag{ArcGIS}
\cvtag{Aimsun}
\cvtag{pandas}
\cvtag{seaborn}
\cvtag{linux}
\cvtag{git}
\cvtag{Vim}



